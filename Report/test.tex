\documentclass[12pt, letterpaper,titlepage]{article}
\usepackage[margin=0.5in]{geometry}
\setlength{\parindent}{10ex}
\usepackage[utf8]{inputenc}
\usepackage{caption}
\usepackage{subcaption}
\usepackage{graphicx}
\usepackage{amsmath}
\usepackage{hanging}
\usepackage[margin=0.5in]{geometry}
\graphicspath{ {figures/} }
\usepackage{appendix}
\usepackage{listings}
\usepackage{hanging}

\newcommand{\horrule}[1]{\rule{\linewidth}{#1}}
\title{	
\normalfont \normalsize 
\textsc{Oklahoma State University} \\ [25pt] 
\textsc{College of Engineering, Architecture and Technology} \\ [25pt]
\textsc{Department of Mechanical and Aerospace Engineering} \\ [25pt]
\horrule{0.5pt} \\[0.4cm] % Thin top horizontal rule
\huge  Mechatronics Design \\ 
\huge  Development of Low Cost Open-Source Robotics Learning Platform\\
\horrule{0.5pt} \\[0.5cm] % Thick bottom horizontal rule
}

\author{Diego Alejadro Colón Serrano}
\date{May 7\textsuperscript{th}, 2020}

\begin{document}

\maketitle
\tableofcontents
\pagebreak

\section{Introduction}

	When looking for a mobile robotic platform to start in the hobby, it is quite daunting to select where to start. If you are completely new to programming and hardware design, a premade kit might seem like a good value proposition. The kit includes step-by-step instructions, all the parts required and there are many tutorials online on how to do things. Now, this option might be too simple for someone with more technical knowledge and is interested in mobile robotics and wants a platform to work with. Is there something that can fit the needs of both of them? Yes, there are but the prices of platforms that are accessible to newcomers but still relevant to experienced individuals are significant investments. For this reason, I set out to create a more budget-friendly mobile robotics platform that can be built and programmed by newcomers and expanded upon as they grow in the hobby. A platform that is simple to put together and work with but allows for possibilities past its original design. With that goal in mind, I designed a platform that achieves that which it's commercial competitors do while reducing the entrance price.

\section{Methods}
	
	Explain in detail your design and methods needed to realize the project, what mechatronics components are necessary, and other relevant details. Also it can include possible limitations, etc. In this part you can make reference to the three papers and other documents you are using as support.

\section{Assumptions and Procedures}
	Discuss andy assumptions you make in the project, the procedures required to complete and demonstrate your project

\section{Results and Discussion}
	Demonstrate your mechatronc system (simulator or experiment). Provide a link to your demonstration. Collect and present necessary dtat to prove the effectiveness of your design

\section{Conclusions and Recommendations}
	Conclusions and recommendations

\section{References}



\end{document}