\documentclass[12pt, letterpaper,titlepage]{article}
\usepackage[margin=0.5in]{geometry}
\usepackage[utf8]{inputenc}
\usepackage{graphicx}
\usepackage{caption}
\usepackage{amsmath}
\usepackage{cite}
\graphicspath{ {figures/} }


\newcommand{\horrule}[1]{\rule{\linewidth}{#1}}
\title{	
\normalfont \normalsize 
\textsc{Oklahoma State University} \\ [25pt] 
\textsc{College of Engineering, Architecture and Technology} \\ [25pt]
\textsc{Department of Mechanical and Aerospace Engineering} \\ [25pt]
\horrule{0.5pt} \\[0.4cm] % Thin top horizontal rule
\huge  Mechatronics Design \\ 
\huge  Development of a Low Cost Open-Source Mobile Robotics Learning Platform\\
\horrule{0.5pt} \\[0.5cm] % Thick bottom horizontal rule
}

\author{Diego Alejadro Colón Serrano}
\date{May 7\textsuperscript{th}, 2020}

\begin{document}

\maketitle
\tableofcontents
\pagebreak

\section{Introduction}

	When looking for a mobile robotic platform to start in the hobby, it is quite daunting to select where to start. If you are completely new to programming and hardware design, a premade kit might seem like a good value proposition. The kit includes step-by-step instructions, all the parts required and there are many tutorials online on how to do things. Now, this option might be too simple for someone with more technical knowledge and is interested in mobile robotics and wants a platform to work with. Is there something that can fit the needs of both of them? Yes, there are but the prices of platforms that are accessible to newcomers but still relevant to experienced individuals are significant investments. For this reason, I set out to create a more budget-friendly mobile robotics platform that can be built and programmed by newcomers and expanded upon as they grow in the hobby. A platform that is simple to put together and work with but allows for possibilities past its original design. With that goal in mind, I designed a platform that achieves that which it's commercial competitors do while reducing the entrance price. \cite{DUMMY:1}

\section{Design}

\subsection{Design objectives}

	Before the design aspect of this project could take place, it was important to establish exactly what the design had to accomplish. For this project, the final concept must satisfy the following criteria:

	\begin{enumerate}
		\item The platform must have a minimal cost.
		\item The platform must accessible to beginners and relevant to advanced users
		\item The platform must allow for additional expandability
		\item The platform must be able to be customized depending on the resources the user has available
		\item The platform must be a tool that allows the users to progress in the field towards more complex projects
	\end{enumerate}

\subsection{Benchmark}

	In order to complete this project a benchmark had to be established. A quick search on Amazon showed several different types of mobile robotics kits. These ranged from barebones kits that provided everything but a microcontroller such as (barebones) to kits that included everything required to get started such as (completes). The variety in entry level robotics kits is amazing and allows users to select something that fits their needs best. And although the complete kits do not offer as much flexibility as the barebones kits, users might favor a complete kit to avoid some of the complications that come with a barebones kit. 

	Currently the standard in mobile robotics platform for research appear to be Turtlebots. Turtlebots are small battery powered differential drive platforms controlled with a Raspberry Pi connected to a base computer. They are used to teach at robotics classes at universities and as platforms for current robotics research. The software that is used to operate them is based on ROS (Robotics Operating System), an open-source project that has been the standard for robotics reserach and applications. This is the standard for which the project will aim.

	The Turtlebot is currently in its third version and sells at a starting MSRP of 549.00 USD for the base model. The base model is equipped with: a Raspberry Pi 3, two encoded motors, 360 degree LiDAR sensor, 11.1V Li-Po battery, IMU unit and a low level control board to interract with the encoders and sensors. This combination of sensors, actuators and computation allows for a very versitile platform on which many different aspects of mobile robotics can be investigated and explored.

\subsection{Mechanical Design}

	For the mechanical design of the project, it was decided to make the structure as simple as possible. This led to the idea of a stacked platform where a battery is protected between two plates. T
	
	The bottom plate consists of a 
	
	he bottom of these plates has a the motors attached to it, while the top plate has all of the electronics attached to it. With the concept of future expandabilityin as one of the design objectives, it was important for the frame to be as open and accessible as possible to the user. All of the components are easily accessible and interchangable. 



\subsection{Electrical Design}

\subsection{Software Design}

\section{Validation}

\subsection{Assumptions}

\subsection{Procedures}
	Discuss andy assumptions you make in the project, the procedures required to complete and demonstrate your project

\section{Results}
	Demonstrate your mechatronc system (simulator or experiment). Provide a link to your demonstration. Collect and present necessary dtat to prove the effectiveness of your design

\section{Discussion}

\section{Conclusions}

\section{Recommendations}

\bibliographystyle{IEEEtran}
\bibliography{research}

\end{document}